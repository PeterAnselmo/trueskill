\section{Analytical solution of the score-based Bayesian rating approach}

\subsection{General}

Our starting point is a game with the outcome $\Delta s$. Involved are $n$ teams and $m$ players. 


The general idea is to find the probability for the skills: 
\begin{equation}
p(\mathbf{l}) = \prod_i \mathcal{N}(l_i;\mu_i,\simga_i)
\end{equation}

The probability  distribution $\mathcal{N}(l_i;\mu_i,\simga_i)$ is estimated from the game outcomes. A game outcome is measured in the score difference $\Delta s$ = score(team1) - score(team2). After every game, the conditional probability 
\begin{equation}
p(\mathbf{l} | \Delta s) 
\end{equation}
is used as the new estimate for the skills.

Using the Bayes theorem, we can write
\begin{equation}
p(\mathbf{l} | \Delta s) \propto p( \Delta s | \mathbf{l} ) p(\mathbf{l})
\end{equation}

This equation can be alternatively stated as a Hidden Markov model using following considerations
\begin{itemize}
\item the player performance $p_i$ is uniquely determined by the player's skill $l_i$
\item the individual players performances determine the overall team performance $t_i$
\item the differences in the team performances are described as the performance differences $d$
\item the performance difference determines the game outcome, i.e. the score difference $\Delta s$
\end{itemize}

Thus, we can look at processes
\begin{equation}
p(\Delta s | d) p(d | \mathbf{t} ) p{\mathbf{t} | \mathbf{p} ) p(\mathbf{p} | \mathbf{l}) p(\mathbf{l})
\end{equation}

Using such intermediate processes, one can describe $p(\mathbf{l} | \Delta s)$ by integrating out the intermediate variables $d$, $\mathbf{t}$ and $\mathbf{p}$. Thus, we get
\begin{equation}
p(\mathbf{l} | \Delta s) = \int_{-\infty}^{\infty} dd \int d^n t \int d^mp p(\Delta s | d) p(d | \mathbf{t} ) p{\mathbf{t} | \mathbf{p} ) p(\mathbf{p} | \mathbf{l}) p(\mathbf{l}).
\end{equation}


\subsection{Probabilities}

The skills $l_i$ are assumed to be independent and Gaussian distributed, i.e.
\begin{equation}
p(\mathbf{l}) = \prod_i \mathcal{N}(l_i;\mu_i,\simga_i)
\end{equation}

The game outcome $\Delta s$ is assumed to arise from the differences in the team performances 
\begin{equation}
p(\Delta s | d ) = \mathcal{N}(s; d, \gamma)
\end{equation}
where $\gamma$ is the variance of the game outcome (should easily be obtained from existing game data). 

In a smilier way, the probability $p(\Delta s | d )$ can be described as 
\begin{equation}
p(\Delta s | d ) = \mathcal{N}(s; d, \beta).
\end{equation}




\subsection{Teams and team performances}

Skills can be additive or non-additive. Additive skills are seen in games like Halo or soccer. In such games, more players in a team lead to a higher overall skill.

Quite contrary is the case in games like several card games. Here, more players does not mean more skill in a team. A good example is the German game Skat. Here, three players compete and, interestingly, the player who plays alone has a higher winning probability. 

As will be seen below, the additive and non-additive skills will be described slightly different. 

The probabilities $ p(d | \mathbf{t} )$ and $ p{\mathbf{t} | \mathbf{p} )$ depend on the time size and I will give examples how to describe them. 

Lets focus on the case of two teams $t_1$ and $t_2$. Then the difference in team performances is simply
\begin{equation}
p(d | \mathbf{t} ) = \delta(d - (t_1-t_2))
\end{equation}
where $\delta(..)$ is the Dirac-delta function. 

How to obtain the team performances will be illustrated on an example:

Example: 1 vs 3 

Team 1 is comprised of player $i$. Team 2 is comprised of players $j$, $k$ and $l$.  Hence, we describe the probability in the case of additive skills:
\being{equation}
p{\mathbf{t} | \mathbf{p} ) = \delta(t_1 - p_i)\delta(t_2 - (t_j + t_k + t_l)).
\end{equation}

In the case of non-additive skills, this writes 
\being{equation}
p{\mathbf{t} | \mathbf{p} ) = \delta(t_1 - p_i)\delta(t_2 - (t_j + t_k + t_l)/3).
\end{equation}

Since all functions are either assumed as Gaussian or Dirace delta functions, one can now explicitly solve the integrals and obtain $p(\mathbf{l} | \Delta s) $. To obtain the individual update $p(l_i | \Delta s)$, it remains to integrate out all other skill variables $l_j$, $j\neq i$.


\subsection{Analytical solution}

A general solution can be found in the case of team 1 vs team 2. To define such a solution in a general way, we need the following parameters:
\begin{itemize}
\item $n_{rm pl}$ is number of players in own team
\item $n_{\rm pop}$ is number of players in opponents team
\item $ n = n_{rm pl} + n_{\rm pop}$ is the overall number of players participating in the game
\item $n_{\rm max} = \max(n_{rm pl},n_{rm pop})$ is the number of players in the largest team (analogously $n_{\rm min}$.
\item $\Delta n = n_{\rm max} - n_{\rm min}$ is the difference between the team sizes. 
\item $\sigma$ and $\mu$ are the standard deviation and mean, respectively, of the player's skill distribution. 
\item $\simga_{\rm pl}^2 $ and $\mu_{\rm pl}$ are the sum of the teams variances and mean values, respectively
\item $\simga_{\rm opp}^2 $ and $\mu_{\rm opp}$ are the sum of the opponent teams variances and mean values, respectively
\item again, $\Delta s = $ score(player) - score(opponent)
\end{itemize}

The update are here presented as precision $\pi = 1/\sigma^2$ and precision adjusted mean $\tau = \mu/\sigma^2$. 


In the case of additive skills, the update is given as 
\begin{equation}
\pi_i^{\rm new} = \frac{1}{\sigma^2} +  \frac{1}{n \beta^2 + \gamma^2 + \sigma_{\rm opp}^2 + \sigma_{\rm pl}^2 - \sigma^2 )
\end{equation}
\begin{equation}
\tau_i^{\rm new} = \frac{\mu_i}{\sigma^2} +  \frac{\Delta s - \mu_{\rm pl} + \mu + \mu_{\rm opp}}{n \beta^2 + \gamma^2 + \sigma_{\rm opp}^2 + \sigma_{\rm pl}^2 - \sigma^2 )
\end{equation}



In the case of non-additive skills we obtain
\begin{equation}
\pi_i^{\rm new} = \frac{1}{\sigma^2} + \frac{1}{n_{\rm pl}} \frac{1}{\frac{2 + \Delta n}{n_{\rm max}} \beta^2 + \gamma^2 + \frac{\sigma_{\rm opp}^2}{n_{\rm opp}} + \frac{\sigma_{\rm pl}^2 - \sigma^2}{n_{\rm pl}}}
\end{equation}
\begin{equation}
\tau_i^{\rm new} = \frac{1}{\sigma^2} + \frac{1}{n_{\rm pl}^2} \frac{\Delta s - (\mu_{\rm pl} - \mu)/n_{\rm pl} + \mu_{\rm opp}/ n_{\rm opp} }{\frac{2 + \Delta n}{n_{\rm max}} \beta^2 + \gamma^2 + \frac{\sigma_{\rm opp}^2}{n_{\rm opp}} + \frac{\sigma_{\rm pl}^2 - \sigma^2}{n_{\rm pl}}}
\end{equation}
 