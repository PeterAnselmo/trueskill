\documentclass{article}
\usepackage{graphicx}
 \usepackage{amsmath} 
 
  
\begin{document}

\author{Markus D\"uttmann}



\section{Analytical Solution of the Score-Based Bayesian Rating Approach}



\subsection{Model}

A score-based Bayesian rating method is used to model the skills of players in a game. The game consists of $n$ teams and $m$ players. It is defined by its outcome $\Delta s$. The aim is to estimate the skill distribution $p(\mathbf{l})$. 

The individual player skills $l_i$ are assumed to be independent characteristics of the player's ability to score
\begin{equation}
p(\mathbf{l}) = \prod_i p(l_i)\, .
\end{equation}

The score-based Bayesian rating method updates its skill estimate after each game according to the outcome. The conditional probability 
\begin{equation}
p(\mathbf{l} | \Delta s) 
\end{equation}
is used as the new estimate for the skills.

Using the Bayes theorem, one can write
\begin{equation}
p(\mathbf{l} | \Delta s) \propto p( \Delta s | \mathbf{l} ) p(\mathbf{l})
\end{equation}

This equation can be alternatively stated as a Hidden Markov model using following considerations
\begin{itemize}
\item the player performance $p_i$ is determined by the player's skill $l_i$
\item the individual players performances determine the overall team performance $t_i$
\item the differences in the team performances are described as the performance differences $d$
\item the performance difference determines the game outcome, i.e. the score difference $\Delta s$.
\end{itemize}

Thus, we can look at processes
\begin{equation}
p(\Delta s | d) p(d | \mathbf{t} ) p(\mathbf{t} | \mathbf{p} ) p(\mathbf{p} | \mathbf{l}) p(\mathbf{l})
\label{process}
\end{equation}
Using these, one can describe $p(\mathbf{l} | \Delta s)$ by integrating out the intermediate variables $d$, $\mathbf{t}$ and $\mathbf{p}$. Thus, we get
\begin{equation}
p(\Delta s | \mathbf{l} | ) = \int_{-\infty}^{\infty} dd \int d^n t \int d^mp~ p(\Delta s | d) p(d | \mathbf{t} ) p(\mathbf{t} | \mathbf{p} ) p(\mathbf{p} | \mathbf{l}) .
\end{equation}
Using (\ref{process}), we obtain the skill estimate
\begin{equation}
p(\mathbf{l} | \Delta s) = \int_{-\infty}^{\infty} dd \int d^n t \int d^mp~ p(\Delta s | d) p(d | \mathbf{t} ) p(\mathbf{t} | \mathbf{p} ) p(\mathbf{p} | \mathbf{l}) p(\mathbf{l}).
\label{integral}
\end{equation}


\subsection{Probabilities}

The skills $l_i$ are assumed to be independent and Gaussian distributed, i.e.
\begin{equation}
p(l_i) =\mathcal{N}(l_i;\mu_i,\sigma_i) \,.
\end{equation}
They are described by their mean $\mu_i$ and the corresponding standard deviation $\sigma_i$. 

The player performance of a player may depend on various things, depending, for instance, on the specific game setup or his condition. Therefore, the performance should be described as a random process described by its probability function 
\begin{equation}
p(p_i | l_i) = \mathcal{N}(p_i; l_i, \beta)\,.
\end{equation}
The standard deviation $\beta$ quantifies the uncertainties in predicting the player's performance. 

Furthermore, the game outcome $\Delta s$ is assumed to be a random process. It arises mainly due to the performance differences of the teams. It can be described as 
\begin{equation}
p(\Delta s | d ) = \mathcal{N}(s; d, \gamma)\, ,
\end{equation}
where $\gamma$ is the variance of the game outcome and should easily be obtained from existing game data. 




\subsection{Teams and Team Performances}

Are more players better all the time? This certainly is the case in games like Halo or even soccer. In such games, more players in a team lead to a higher overall skill. Thus, the skills are additive. 

Quite contrary is the case in certain games, e.g. card games like Skat or Briddge. Here, more players do not necessarily lead to a stronger team. A good example is indeed the German game Skat. Here, three players compete and, interestingly, the player who plays alone has a higher winning probability. In such cases, the skills are non-additive and teams must be described in a different way. 

The specific game setup is defined by the teams. They are encoded in the the probabilities $ p(d | \mathbf{t} )$ and $ p(\mathbf{t} | \mathbf{p} )$, which depend on the time size on the nature of the game (additive or non-additive). Below, I will give examples how to obtain these probability functions. 

Lets focus on the case of two teams $t_1$ and $t_2$. Then the difference in team performances is simply $d = t_1 - t_2$. The corresponding probability densities are
\begin{equation}
p(d | \mathbf{t} ) = \delta[d - (t_1-t_2)]
\end{equation}
where $\delta(..)$ is the Dirac-delta function. 

The team performances depend on the team members. In the following example, the case "1 vs 2" will be discussed:

Team 1 is comprised of player $i$. Team 2 is comprised of players $j$, $k$ and $l$.  Hence, we describe the probability in the case of additive skills as $t_1 = p_i$ and $t_2 = t_j + t_k + t_l$. Accordingly, we have
\begin{equation}
p(\mathbf{t} | \mathbf{p} ) = \delta(t_1 - p_i)\delta[t_2 - (t_j + t_k + t_l)].
\end{equation}

In the case of non-additive skills, the average skills are important. Hence, the team performances are  $t_1 = p_i$ and $t_2 = (t_j + t_k + t_l)/3$ and the probability densities are
\begin{equation}
p(\mathbf{t} | \mathbf{p} ) = \delta(t_1 - p_i)\delta[t_2 - (t_j + t_k + t_l)/3].
\end{equation}

The aim is to calculate the integral (\ref{integral}). All factors are known and, fortunately, they are either Gaussians or Dirace delta functions. Therefore, one can explicitly solve (\ref{integral}) and obtain $p(\mathbf{l} | \Delta s) $ as a new estimate for the player's skills. To obtain the estimate for a single skill $p(l_i | \Delta s)$, it remains to integrate out all other skill variables $l_j$, $j\neq i$.


\subsection{Analytical solution}

A general solution can be found in the case of team 1 vs team 2. To define such a solution in a general way, we need the following parameters:
\begin{itemize}
\item $n_{\rm pl}$ is number of players in own team
\item $n_{\rm opp}$ is number of players in opponents team
\item $ n = n_{\rm pl} + n_{\rm opp}$ is the overall number of players participating in the game
\item $n_{\rm max} = \max(n_{\rm pl},n_{rm pop})$ is the number of players in the largest team (analogously $n_{\rm min}$.
\item $\Delta n = n_{\rm max} - n_{\rm min}$ is the difference between the team sizes. 
\item $\sigma$ and $\mu$ are the standard deviation and mean, respectively, of the player's skill distribution. 
\item $\sigma_{\rm pl}^2 $ and $\mu_{\rm pl}$ are the sum of the teams variances and mean values, respectively
\item $\sigma_{\rm opp}^2 $ and $\mu_{\rm opp}$ are the sum of the opponent teams variances and mean values, respectively
\item again, $\Delta s = $ score(player) - score(opponent)
\end{itemize}
The new skill estimate is here represented as precision $\pi = 1/\sigma^2$ and precision adjusted mean $\tau = \mu/\sigma^2$. 


In the case of additive skills, the update is given as 
\begin{equation}
\pi_i^{\rm new} = \frac{1}{\sigma^2} +  \frac{1}{n \beta^2 + \gamma^2 + \sigma_{\rm opp}^2 + \sigma_{\rm pl}^2 - \sigma^2 }
\end{equation}
\begin{equation}
\tau_i^{\rm new} = \frac{\mu_i}{\sigma^2} +  \frac{\Delta s - \mu_{\rm pl} + \mu + \mu_{\rm opp}}{n \beta^2 + \gamma^2 + \sigma_{\rm opp}^2 + \sigma_{\rm pl}^2 - \sigma^2 }
\end{equation}



In the case of non-additive skills we obtain
\begin{equation}
\pi_i^{\rm new} = \frac{1}{\sigma^2} + \frac{1}{n_{\rm pl}} \frac{1}{\frac{2 + \Delta n}{n_{\rm max}} \beta^2 + \gamma^2 + \frac{\sigma_{\rm opp}^2}{n_{\rm opp}} + \frac{\sigma_{\rm pl}^2 - \sigma^2}{n_{\rm pl}}}
\end{equation}
\begin{equation}
\tau_i^{\rm new} = \frac{1}{\sigma^2} + \frac{1}{n_{\rm pl}^2} \frac{\Delta s - (\mu_{\rm pl} - \mu)/n_{\rm pl} + \mu_{\rm opp}/ n_{\rm opp} }{\frac{2 + \Delta n}{n_{\rm max}} \beta^2 + \gamma^2 + \frac{\sigma_{\rm opp}^2}{n_{\rm opp}} + \frac{\sigma_{\rm pl}^2 - \sigma^2}{n_{\rm pl}}}
\end{equation}

\end{document}

 